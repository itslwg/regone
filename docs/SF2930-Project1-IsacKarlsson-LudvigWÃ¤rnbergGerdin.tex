% Created 2020-03-05 Thu 14:30
% Intended LaTeX compiler: pdflatex
\documentclass[11pt]{article}
\usepackage[utf8]{inputenc}
\usepackage[T1]{fontenc}
\usepackage{graphicx}
\usepackage{grffile}
\usepackage{longtable}
\usepackage{wrapfig}
\usepackage{rotating}
\usepackage[normalem]{ulem}
\usepackage{amsmath}
\usepackage{textcomp}
\usepackage{amssymb}
\usepackage{capt-of}
\usepackage{hyperref}
\usepackage{minted}
\usepackage[margin=1.25in]{geometry} \usepackage{booktabs} \usepackage{graphicx} \usepackage{adjustbox} \usepackage{amsmath} \hypersetup{colorlinks=true,linkcolor=blue} \usepackage{amsthm} \newtheorem{definition}{Definition} \usepackage{bookmark}
\author{Ludde}
\date{\today}
\title{}
\hypersetup{
 pdfauthor={Ludde},
 pdftitle={},
 pdfkeywords={},
 pdfsubject={},
 pdfcreator={Emacs 26.2 (Org mode 9.1.14)}, 
 pdflang={English}}
\begin{document}

\begin{titlepage}
\centering
\includegraphics[width=0.15\textwidth]{example-image-1x1}\par\vspace{1cm}
{\scshape\LARGE Kungliga Tekniska Högskolan \par}
\vspace{1cm}
{\scshape\Large SF2930 Regression Analysis \par}
\vspace{1.5cm}
{\huge\bfseries Report I \\  \par}
\vspace{2cm}
{\Large\itshape Isac Karlsson\\ Ludvig Wärnberg Gerdin}
\vfill
Examiner \par
\textsc{Tatjana Pavlenko}

\vfill

{\large \today\par}
\end{titlepage}

\newpage
\tableofcontents
\newpage

\section{Introduction and Project Goals}
\label{sec:org7a257c2}
\subsection{Introduction}
\label{sec:org790e813}
Our choice of scenario is Scenario I: Body fat assessment, which involves Large-Sample regression (p < n). 
According to the World Health organization (WHO) obesity, the state where excess body fat is causing
extensive health effects, is a large risk factor for some chronic diseases. Some examples are cancer
and diabetes. Since the number of cases of obesity is increasing one may want to identify these people 
quickly and reliably.

\subsection{Data Description}
\label{sec:orga503f87}

Since BMI has shown to be a bad predictor of actual fatness, this project focuses on body fat mass (BFM).
There exists very accurate methods for calculating BFM but because of high costs and efforts cheaper 
methods such as regression models are widely used. 

The given dataset (BFM MEN) describes data of body density (calculated using underwater weighing), 
age and other anthropometric variables about 252 men.

\subsection{Project Goals}
\label{sec:orga735cdd}

The main goal of the project is to create and validate our own regression model in order to predict BFM.
This includes the following:

\begin{enumerate}
\item Residual analysis for model adequacy checking
\item Handling of outliers, influential observations and leverage
\item Transformations of variables in order to correct model inadequacies
\item Multicollinearity treatments and diagnostics
\item Different types of variable selection and evaluation of these using cross validation
\item Computer-intensive procedures for model assessment (e.g. bootstrap residuals)
\end{enumerate}

\newpage
\section{Analyses and Model Development}
\label{sec:org4a56173}

The information presented in the proceding sections are primarily taken from \textit\{Introduction to
Linear Regression Analysis\} \cite{Montgomery2012}. If not, the information is cited.

\subsection{Residual analysis}
\label{sec:orgb4bdb08}

Some major assumptions we use in our analysis are:

\begin{enumerate}
\item The errors \(\epsilon_i\) for observation \(i\) are independently and identically normally distributed.
\item Mean of \(\epsilon = 0\)
\item Variance of \(\epsilon = \sigma^2\), where \(\sigma\) is a constant.
\item There is approximately a linear relationship between the regressors and the response (\(y\)).
\end{enumerate}

When analysing violations of the assumptions given above, the primary tool is using the model residuals. 
We define the residual for observation \(i\) as

\[
   e_i = y_i - \hat{y_i}, \ i = 1, ... , n
   \]

One may view a residual as the difference between the data and the fit although it is also a way to analyze 
the variability in the response variable that cannot be explained by the regression model. Plotting residuals
is a effective method to examine how the regression model fits the data and make sure the assumptions listed 
are not violated.

\subsubsection{R-Student}
\label{sec:org71888cd}

One type of residual is the externally studentized residual, which is given by

\[
    t_i = \frac{e_i}{\sqrt{S^2_{i}(1 - h_{ii})}}, \ i = 1, ..., n
    \]

The externally studentized residual is also called the R-student residual. 
Here an estimate of \(\sigma^2\) is used instead of \(MS_{Res}\) in order to create an 
\textit{externally} studentized residual.

Now we introduce some basic residual plots, which are commonly generated using computers. These
should be analyzed routinely when solving any kind of regression modelling problem. Note that the
externally studentized residuals are often the ones plotted since they have constant variance.

\subsubsection{Normality of residuals}
\label{sec:org303dc61}

This is a tool for analysing if two datasets (of quantiles) come from the same probability distribution. 
By plotting the quantiles against each other we will hopefully see somewhat of a straight line. This 
corresponds to them originating from the same distribution. 

Here some small departures from the normality assumption does not have a large impact. Meanwhile 
large nonnormality could have more of an impact on the regression modelling process. Mainly, the problems 
relate to inference of the model building - for example, prediction intervals depend on the 
normality assumption. One may check the normality assumption simply by constructing a normal probability
plot of the residuals. 

\subsubsection{Fitted Values Against Residuals}
\label{sec:org44372a9}

Simply a plot of the, often externally studentized, residuals versus the fitted values. This is useful
because it allows an easy way to detect model inadequacies. If the plot shows the residuals contained in
a horizontal band, then the model does not contain any obvious defects. If this is not the case one may
conclude that there are likely model imperfections.

\subsubsection{Added Variable Analysis}
\label{sec:org1058b29}

Particularly useful when analysing if the relationship between the regressor variables and the response
has been defined accurately. Another way to use these plots are when evaluating the marginal usefulness
of some variable that is not presently a part of the model. Here \(y\) (the response variable) and \(x_j\)
(regressor) is regressed against the regressors (currently present in the model) and the residuals that
follow for each regression. When plotting these residuals against each other one may analyse the marginal
relationship for the regressor \(x_j\) that has caught our attention.

\subsection{Diagnostics and handling of Outliers}
\label{sec:org60f2d42}
\subsubsection{Treatment of outliers}
\label{sec:orgec4b65b}

An observation that is noticeably different from the rest of the data is considered an outlier. A way
to spot y space outliers is simply by analyzing the residuals. The ones that are noticeably larger 
(when considering the absolute value of these residuals) than the other residuals is an indication of
potential outliers. The magnitude of the impact caused by these outliers depends on their location
in x space. An example of identifying potential outliers is by using scaled residuals (e.g. R-student). 

Note that outliers that are considered bad values, e.g. values from mis-measuresments,
should preferably be discarded. Meanwhile there should
always be non-statistical confirmation that the outlier really is a bad value before discarding it. One
could argue that outliers are the most important part of the data since it often control many 
properties when modelling. 

One way to analyse the effect of each outliers is by simply not including the data point and refitting.
In general we prefer it when the model is not too sensitive to a small number of observations. 

The hat matrix is can be very useful when detecting potential outliers, since it determines the variances
and covariances of \(\hat{y}_j\) and \(\textbf{e}\). Each element \(h_{ij}\) corresponds to the amount of
leverage exercised by the ith observation \(y_i\) on the jth, fitted value, \(\hat{y_j}\).

It appears that large hat diagonals may correspond to an influential outlier since they are remote
in x space when compared to the rest of the data. Knowing this analysts also want to observe
the studentized residuals of each observation. Large hat diagonals along with large residuals 
are likely an influential observation. 

\subsubsection{Cook's Distance}
\label{sec:org026f396}

One way to both of these at the same time is by using the squared distance between the least-squares
estimate (based on all n points) and also the estimate obtained when deleting the ith point. This is
called Cook’s distance and can be interpreted as the euclidean distance that the vector containing fitted
values is moved when deleting the ith observation. 

The Cook's distance is arguably one of the more important metrics for our prediction purpose, since is highlight's
the observation's effect on the predicted y-values. \cite{22286}

\subsubsection{DFFITS \& DFBETAS}
\label{sec:org1b1aefb}

Two other measures of the effects when deletion an observation is \(DFBETAS\) and \(DFFITS\). \(DFBETAS\) tells us
about the effects on the regression coefficient \$\hat{\beta_j} when deleting the ith observation. It is defined as
follows and is given in units of standard deviation.

\(DFFITS\) analyses the effects on the fitted value when deleting the ith observation. Here \(DFFITS\) tells us
the number of standard deviations that the fitted value is changed by when deleting observation \(i\). Since 
the \(DFFITS\) values consider the effect on the fitted value, this metric is arguably one of the more important 
ones for our purpose.

\(DFBETA\) is presumably more interesting from an explanatory point-of-view \cite{22286}, which is not the
primary purpose of this report. We therefore analyse the Cook's distance and the \(DFFITS\) values more
thoroughly that the \(DFBETA\) values.

\subsection{Transformations of variables}
\label{sec:org9906d91}

Whenever an assumption mentioned above is violated it is usually a good idea to consider data transformation. 
In some cases expressing the regressor and or the response variables using another measurement results in 
violations no longer being present, e.g. inequality of variance. 

If we wish to transform y, in order to correct for example nonconstant variance, we can use the power
transformation ylambda where lambda is what we want to determine. We can do this by using the Box-Cox method
which also allows us to estimate the parameters of the regression model simultaneously, using maximum likelihood.
The method is described as follows:

Note that when analysing a partial regression plot for some regressor variable x1 ,entering the model linearly, 
then partial residuals will show a straight line. Note that the slope of this line is the regression coefficient 
of x1 in the multiple regression model. When x1 is considered a candidate variable for the model, if the partial
regression plot shows a horizontal band, that tells us that no additional information for predicting y is 
described by x1. When the partial regression plot shows a curvilinear band, then one may use a transformation 
(e.g. replacing x1 with 1/x1).

\subsection{Diagnostics and handling of Multicolinearity}
\label{sec:org6a27468}

Note that if the equation given above is approximately true, at least for some subset of the columns of X,
then the problem of multicollinearity exists. As a result of this the least-squares analysis, of the model
itself, may be very deficient. This may cause the usefulness of the regression model to decrease significantly. 

One simple way to detect multicollinearity is by inspecting the off-diagonal element rij in X’X. A near
linear dependency between xi and xj will result in abs(rij) to be near unity. Note that this is useful for
detecting linear dependence between pairs of regressors and that this can not be used as a tools for
detecting anything more complex than that. Therefore, this method of detecting multicolinearity will
only be considered as a complementary method to more appropriate methods described here.

The diagonal elements of the matrix C = (X’X)-1 can also be used for detecting multicollinearity. Note that 
the jth element of C can be written as follows: Cjj=(1-Rj2)-1, 
here R\(^{\text{2j}}\) is obtained when xj is regressed on the other p-1 regressors.
When xj is almost orthogonal to the other regressors, Rj2 is small and Cjj is close to unity. Meanwhile 
if xj is nearly linear dependent, on a subset of the other regressors, R2j is close to unity and Cjj is large.

One may also analyze the characteristic roots/eigenvalues of X’X to measure the extent of multicollinearity. 
When one or more of the eigenvalues are small, then there exists one or more near-linear dependencies. 
The condition number of X’X defined as:

if <100, no serious problem
if between 100-1000, medium multicollinearity
if >1000, strong multicollinearity

As an ending note, we should mention the inhererent multicolinearity in this dataset. Most candidate predictors 
are measures of body size, which naturally causes the predictors to be linearly related in to each other. That 
being said, it is still appropriate to investigate methods to alleviate the effect of multicolinearity since 
the stability of the model is heavily influenced by it. 

\subsection{Computer-Intensive Procedures and Variable Selection}
\label{sec:orgb254a1f}
\subsubsection{The Boostrap}
\label{sec:org4845e4f}
Bootstrapping is a computer-intensive technique that allow us to compute reliable estimates of the standard
errors of regression estimates when there is no standard procedure available or cases where the results are
only approximate techniques (e.g. based on large-sample theory). 

If we are interested in a particular regression coefficient BetaHat. First we are required to select a random
sample of size n with replacement from this original sample, this is called the bootstrap sample. Then we
proceed to fit the model to this sample by using the procedure as for the original sample. This gives us
the first bootstrap estimate BetaHat1(star). We repeat this process many times and each repetition, a new 
bootstrap sample is selected, the model is fit, and an estimate BetaHati(star) is concluded. 
\subsubsection{Variable selection}
\label{sec:org7e9bebb}
If multicollinearity is present, variable selection methods are very useful. Note that variable selection does
not result in complete elimination of multicollinearity, in some cases two or more regressors are highly related 
even though some subset of them indeed should be a part of the model, instead it helps us justify the presence
of multicollinearity in the final model. One should also note that experience and subjective considerations
should always be considered as a part of the variable selection problem.

\subsubsection{All Possible Regression and Other Methods}
\label{sec:orgfb58300}

Simply requires to fit all the regression equations starting with one candidate regressor, then two
candidate regressors and so on. These are later analyzed regarding some criterion and the “best” one is selected. 

Since evaluating all possible regressions can sometimes be time consuming computationally, there are other
methods for evaluating only a smaller number of subset regression models by adding/removing regressors one
at a time. These methods are generally called stepwise procedures, and examples are forward selection and backward
elimination. These are not considered here, since the use of all possible regression is justified.

Note that we have not included any of the stepwise regression methods mentioned above. Primarily
because of the list of problems connected with these methods \cite{20856}, which are for example that they yield
R-squared values that are highly biased and cause severe problems in the presence of collinearity.

\newpage
\section{Results}
\label{sec:org50155dd}
\subsection{Sample characteristics}
\label{sec:org8f39d8f}

Table \ref{tab:tblone} reports the sample characteristics. These are left for the reader, in particular to
compare with the outliers presented in section \ref{sec:orgf5bd8bc}.

\input{../main/tblone.tex}

\subsection{Residual analysis}
\label{sec:org58ca713}
\subsubsection{Normality of residuals}
\label{sec:org9f40da6}

Figure \ref{fig:orgf03d55e} illustrates a quantile-quantile plot of the externally studentized residuals.
The observer may say that the points exhibit a pattern that indicates that the residuals are distribute with
heavier tails than that of a normal distribution. \cite{Montgomery2012}. Still, the deviations from the
diagonal line is relatively small, and hence we conclude that the residuals are normally distributed.

\begin{figure}[htbp]
\centering
\includegraphics[width=8cm]{/home/ludvigwgerdin/courses/Regression Analysis/regone/main/qqplot.png}
\caption{\label{fig:orgf03d55e}
Normality plot of residuals.}
\end{figure}

\subsubsection{Fitted Against Residuals}
\label{sec:orgd09f6d3}

Figure \ref{fig:org53c733b} illustrates the fitted values \(\hat y_j\) against the R-student residuals. No apparent 
pattern is formed by the points, i.e. the points seem to be randomly scattered along the dotted horizontal
line. Hence we conclude that the residuals have constant variance, and thus assume that the errors do
as well.

\begin{figure}[htbp]
\centering
\includegraphics[width=8cm]{/home/ludvigwgerdin/courses/Regression Analysis/regone/main/far.png}
\caption{\label{fig:org53c733b}
Fitted values against R-student residuals.}
\end{figure}

\subsubsection{Added Variable Analysis}
\label{sec:org5bd47fb}

Partial regression plots are found in figure \ref{fig:orgd5f43f4}, \ref{fig:orgc49f163},
\ref{fig:orgad73196}, and \ref{fig:orgb796e66}. All figures exhibit potential points 
that are unusually large in the x-space and hence their influence on the model fit should be 
examined further. This will be considered in section \ref{sec:org60f2d42}.
Interestingly, 

\begin{figure}[htbp]
\centering
\includegraphics[width=8cm]{/home/ludvigwgerdin/courses/Regression Analysis/regone/main/biceps_forearm_wrist_av.png}
\caption{\label{fig:orgd5f43f4}
Partial regression plots of regressors \texttt{biceps}, \texttt{forearm}, and \texttt{wrist}.}
\end{figure}   

\begin{figure}[htbp]
\centering
\includegraphics[width=8cm]{/home/ludvigwgerdin/courses/Regression Analysis/regone/main/thigh_knee_ankle_av.png}
\caption{\label{fig:orgc49f163}
Partial regression plots of regressors \texttt{thigh}, \texttt{knee}, and \texttt{ankle}.}
\end{figure}

\begin{figure}[htbp]
\centering
\includegraphics[width=8cm]{/home/ludvigwgerdin/courses/Regression Analysis/regone/main/age_weight_height_neck_av.png}
\caption{\label{fig:orgad73196}
Partial regression plots of regressors \texttt{age}, \texttt{weight}, \texttt{height}, and \texttt{neck}.}
\end{figure}

\begin{figure}[htbp]
\centering
\includegraphics[width=8cm]{/home/ludvigwgerdin/courses/Regression Analysis/regone/main/chest_abdomen_hip_av.png}
\caption{\label{fig:orgb796e66}
Partial regression plots of regressors \texttt{chest}, \texttt{abdomen}, and \texttt{hip}.}
\end{figure}

\subsection{Significance tests}
\label{sec:orgddccd55}

Table \ref{tab:anova} presents the Analysis of Variance table (ANOVA) for the full model. In the 
preceding sections we concluded that the R-student residuals seem to be randomly scattered and 
that the R-student residuals approximately follows a normal distribution. Therefore, we assume 
that the significance tests presented here are valid. 

The results from the ANOVA analysis will not be covered in detail in the preceding sections. Since
our primary purpose is prediction, not explanation, the results presented here are left for the 
readers interpretation. Instead, we place greater emphasis on handling multicolinearity 
(see section \ref{sec:orga21edd2}) and conducting
cross-validation for model development (see section \ref{sec:org7e9bebb}),
since these aspects affect the stability of our predictions and generalizability of our model.

\input{../main/anova.tex}

\subsection{Transformations of variables}
\label{sec:org323e071}

In section \ref{sec:orgb4bdb08} we noted that there was no indication that a transformation was needed on the 
response variable. Here, we will see that the transformation of the response variable skews the results negatively.
Figure \ref{fig:orgd176955} displays the values of \(\lambda\) to be used in a potential Box-Cox transformation of 
the dependent variable \texttt{density}. The \(\lambda\) that maximized the log-likelihood is 0.9 
(0.7-1.1 approximate 95\% CI). Using \(\lambda = 0.9\) gives us the normal probability plot displayed on the 
right hand side in figure \ref{fig:orgd176955}. We notice that this affects the distribution of residuals by
making it more light-tailed. 

\begin{figure}[h]
\centering
\includegraphics[width=8cm]{/home/ludvigwgerdin/courses/Regression Analysis/regone/main/boxcox_fit.png}
\caption{\label{fig:orgd176955}
Values for lambda against the log-likelihood for Box-Cox transformations.}
\end{figure}

\subsection{Diagnostics and Handling of Multicolinearity}
\label{sec:orga21edd2}

Table \ref{tab:mc} presents the VIF for each respective regressor and eigen values of the
\(\textbf{X}\textbf{X}'\). The eigen values for the 
\texttt{biceps}, \texttt{forearm}, and \texttt{wrist} regressors are relatively close to zero, and the
VIF of the \texttt{weight}, \texttt{chest}, \texttt{abdomen}, and \texttt{hip} regressors are larger than 10.
Hence, there appears to be multicolinearity in the data.

A correlation matrix for the full model is found in section \ref{sec:org703fbd5}. The strong collinearity
between the \texttt{weight} regressor and other predictors is apparent in the correlation matrix in figure
\ref{fig:org48be7d5}. The \texttt{weight} regressor shows a strong correlation with all but the \texttt{age} and
the \texttt{height} regressors.

\input{../main/mc.tex} 

In order to handle the multicollinearity in the data, we replace the variables that appear to be involved 
in the multicolinearity with a summary variable. \cite{Montgomery2012} The summary variable is named
\texttt{combo} that were defined as \(\frac{\texttt{hip}\times\texttt{thigh}\times\texttt{abdomen}}{\texttt{weight}}\)
The rationale for this particular variable was that it minimizes the MSE, as well as makes sure that the VIF
are below 10 and that the eigen values of the \(\textbf{X}\textbf{X}'\). The resulting VIF are presented in figure \ref{fig:org04a3053}. We now note that none of the VIF exceed 10.

Now that we've removed several predictors and replaced in with a summary variable, we have conduct the 
residual analysis and observe the effect on the analysis. The plots are presented in \ref{sec:org2c82b23}. We note that 
the effort to reduce multicolinearity did not affect the other diagnostics in a noticeable way. Therefore,
we keep the summary variable and move to handling of outliers.

\begin{figure}[htbp]
\centering
\includegraphics[width=8cm]{/home/ludvigwgerdin/courses/Regression Analysis/regone/woinfluence/vif.png}
\caption{\label{fig:org04a3053}
Variance Inflation Factors (VIF) when using the summary variable \texttt{combo}.}
\end{figure}   

\subsection{Diagnostics and Handling of Outliers}
\label{sec:orgf5bd8bc}

Figure \ref{fig:org37783fd} illustrates Cook's distance for all points, where the three observations with the largest 
Cook's distance are labelled. Considering the cut-off \(D_i = 1\) as proposed in \cite{Montgomery2012}, 
where \(D_i\) is the Cook's distance for observation \(i\), we note that none of the observations would be 
considered influential. Still, observation 39, 83, and 41 are large relative
to the other points in terms of their Cook's distance, which have been mentioned as a diagnostic for further
inspection of outliers. \cite{Fox1991} The three points are henceforth considered as outliers that may
influence our model fit in a considerable way.

\begin{figure}[h]
\centering
\includegraphics[width=8cm]{/home/ludvigwgerdin/courses/Regression Analysis/regone/combo/cd.png}
\caption{\label{fig:org37783fd}
Cook's distance for all observations.}
\end{figure}

Figure \ref{fig:orgab48e78} reports the \(DFFITS\) values. 
The recommended cutoff-value mentioned in \cite{Montgomery2012}, i.e. \(\pm 2\sqrt{\frac{p}{n}}\)
where \(p = 13\) is the number of potential regressors and \(n = 248\) is the sample size, is 
plotted as a dotted line, and the points that lie below or above this cut-off value is labelled.
We observe that several points are considered influential points when using that cut-off value.
The same values as from the Cook's distance plot appear in the \(DFFITS\) plot, but also several 
new points

\begin{figure}[h]
\centering
\includegraphics[width=8cm]{/home/ludvigwgerdin/courses/Regression Analysis/regone/combo/dffits.png}
\caption{\label{fig:orgab48e78}
\(DFFITS\) for all observations.}
\end{figure}

Figure \ref{fig:org6a675ee}, \ref{fig:org5677687}, \ref{fig:org6ba0bbd}, and
\ref{fig:org5aa1ab7} in section \ref{sec:orga908813} presents \(DFBETA\) values for groups of regressors. 
Observation 39 is present in a number of these figures, as well as observation number 83 and 217. 
Using the aforementioned cut-off value of \(\frac{2}{\sqrt{n}}\), we note however that none of these points
would be considered influential points.

We present the observations noted in the Cook's distance and DFFITS plots in Table \ref{tab:influence}.
The points labelled in the \(DFBETA\) plots are not considered by the reason noted previously 
in section \ref{sec:org1b1aefb}. The points that was identified as potential outliers in the added-variable
plots can be compared to the points that are considered as influential in the Cook's distance plots
and the DFFITS plot. For example, we see that observation 39 would be noted as an outlier in a number of 
added-variable plots, and is also in included as one of the more influential observations considering 
its DFFITS and Cook's distance values.

We analyse these observations from two perspectives: Cause of outlier tendencies and effect on fit of 
the model. Looking at the observations, we note that some observations are unlikely but 
still plausible measurements, for example observation 39. In other words, they are likely not results
of mis-measurements, and hence should not be removed for that reason. The second perspective is handled 
section \ref{sec:org7e9bebb}.

\input{../combo/influence_table.tex}

\subsection{Variable selection}
\label{sec:org1be0e81}

The measurements for BIC, the C(p) criterion, and adjusted \(R^2\) of the best subset models are presented
in figure \ref{fig:org2297ae1}. The most well performing model, determined by its cross-validated 
mean squared error, its predictors and the corresponding coefficients along with 95\% confidence intervals are 
presented in Table \ref{tab:coeffs}. The cross-validated MSE for the full model, the model with a summary variable, 
and the model the summary variable without the influential observations are presented in table
\ref{tab:performance}. 

Several methodological considerations were made in this step. Firstly, regarding influential and outlier 
observations. By removing influential observations we reduce the mean squared error by a considerable amount.
However, we have no quantitative nor qualitative reason for removing them. Therefore, we will leave the 
outliers in the dataset. 

Secondly, regarding our method of handling multicolinearity. Since our primary purpose was prediction, 
one could argue that we should proceeded with the model that minimizes the MSE on the test sample, that is
the full model without the summary variable. We would argue, however, that by handling multicolinerity we 
ensure more stable least-squares estimators for the model, and hence more stable predictions. In doing so,
we sacrifice a gain in MSE. There are other methods of handling multicolinearity that were not considered
here, for example Principal Component Regression (PCR) or ridge regression.

Thirdly, the choice to bootstrap confidence intervals around the model coefficients. Another method 
would be conduct the percentile bootstrap on the prediction intervals \cite{davison_hinkley_1997}. This 
would arguably be more useful for our purpose, since the primary use of this model would be prediction. 
However, the CI boostrap around the regression coefficients give us a confidence estimate around 
the coefficients of our model and is therefore useful for prediction as well.

\input{../performance.tex}

\input{../woinfluence/coeffs.tex}

\begin{figure}[htbp]
\centering
\includegraphics[width=8cm]{/home/ludvigwgerdin/courses/Regression Analysis/regone/woinfluence/cv_apr.png}
\caption{\label{fig:orge54577e}
Cross-validated mean squared error for the best subset model and number of regressors.}
\end{figure}

\begin{figure}[htbp]
\centering
\includegraphics[width=8cm]{/home/ludvigwgerdin/courses/Regression Analysis/regone/woinfluence/apr.png}
\caption{\label{fig:org2297ae1}
Number of regressors against multiple performance measures for the best subset regression models.}
\end{figure}

\newpage
\section{Conclusion}
\label{sec:orgd613bc0}

A preliminary model should include. The included predictors and the corresponding coefficients are 

Contextual dimensions are missing for us to make an adequate decision on whether this model should be implemented
such as the cost and context of measurement.

\section{Appendix A}
\label{sec:org703fbd5}

\begin{figure}[H]
\centering
\includegraphics[width=.9\linewidth]{/home/ludvigwgerdin/courses/Regression Analysis/regone/main/hm.png}
\caption{\label{fig:org48be7d5}
Correlation matrix of the full model}
\end{figure}

\newpage

\section{Appendix B}
\label{sec:org2c82b23}
\section{Appendix C}
\label{sec:orga908813}

\begin{figure}[H]
\centering
\includegraphics[width=8cm]{/home/ludvigwgerdin/courses/Regression Analysis/regone/main/biceps_forearm_wrist_dfbeta.png}
\caption{\label{fig:org6a675ee}
\(DFBETA\) for regressors \texttt{biceps}, \texttt{forearm}, and \texttt{wrist}.}
\end{figure}

\begin{figure}[H]
\centering
\includegraphics[width=8cm]{/home/ludvigwgerdin/courses/Regression Analysis/regone/main/thigh_knee_ankle_dfbeta.png}
\caption{\label{fig:org6ba0bbd}
\(DFBETA\) for regressors \texttt{thigh}, \texttt{knee}, and \texttt{ankle}.}
\end{figure}

\begin{figure}[H]
\centering
\includegraphics[width=8cm]{/home/ludvigwgerdin/courses/Regression Analysis/regone/main/age_weight_height_neck_dfbeta.png}
\caption{\label{fig:org5677687}
\(DFBETA\) for regressors \texttt{age}, \texttt{weight}, \texttt{height} and \texttt{neck}.}
\end{figure}

\begin{figure}[H]
\centering
\includegraphics[width=8cm]{/home/ludvigwgerdin/courses/Regression Analysis/regone/main/chest_abdomen_hip_dfbeta.png}
\caption{\label{fig:org5aa1ab7}
\(DFBETA\) for regressors \texttt{chest}, \texttt{abdomen}, and \texttt{hip}.}
\end{figure}

\newpage

\section{References}
\label{sec:orgd857a6a}

\bibliographystyle{plain}
\bibliography{library}
\end{document}