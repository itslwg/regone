% Created 2020-02-13 Thu 16:33
% Intended LaTeX compiler: pdflatex
\documentclass[11pt]{article}
\usepackage[utf8]{inputenc}
\usepackage[T1]{fontenc}
\usepackage{graphicx}
\usepackage{grffile}
\usepackage{longtable}
\usepackage{wrapfig}
\usepackage{rotating}
\usepackage[normalem]{ulem}
\usepackage{amsmath}
\usepackage{textcomp}
\usepackage{amssymb}
\usepackage{capt-of}
\usepackage{hyperref}
\usepackage{minted}
\usepackage[margin=1.25in]{geometry} \usepackage{booktabs} \usepackage{graphicx} \usepackage{adjustbox} \usepackage{amsmath} \hypersetup{colorlinks=true,linkcolor=blue} \usepackage{amsthm} \newtheorem{definition}{Definition} \usepackage{bookmark}
\author{Ludde}
\date{\today}
\title{}
\hypersetup{
 pdfauthor={Ludde},
 pdftitle={},
 pdfkeywords={},
 pdfsubject={},
 pdfcreator={Emacs 26.2 (Org mode 9.1.14)}, 
 pdflang={English}}
\begin{document}

\begin{titlepage}
\centering
\includegraphics[width=0.15\textwidth]{example-image-1x1}\par\vspace{1cm}
{\scshape\LARGE Kungliga Tekniska Högskolan \par}
\vspace{1cm}
{\scshape\Large SF2930 Regression Analysis \par}
\vspace{1.5cm}
{\huge\bfseries Report I \\  \par}
\vspace{2cm}
{\Large\itshape Isac Karlsson\\ Ludvig Wärnberg Gerdin}
\vfill
Examiner \par
\textsc{Tatjana Pavlenko}

\vfill

{\large \today\par}
\end{titlepage}

\newpage
\tableofcontents
\newpage

\section{Introduction and Project Goals}
\label{sec:org0bfdbac}
\section{Analyses and Model Development}
\label{sec:org68d6ee7}
\subsection{Residual analysis}
\label{sec:org3450992}
\subsubsection{Normality of residuals}
\label{sec:orgd303dcc}

The normality of residuals therefore ensures that the confidence intervals presented in section \ref{sec:org06a3ff9}
are valid.

\subsubsection{Fitted Against Residuals}
\label{sec:org4d205a2}
\subsubsection{Added Variable Analysis}
\label{sec:orga5578fd}
\subsection{Diagnostics and handling of Outliers}
\label{sec:org08cae6f}
\subsection{Transformations of variables}
\label{sec:org63c69f0}
\subsection{Diagnostics and handling of Multicolinearity}
\label{sec:orga5f5466}
\section{Results}
\label{sec:org06a3ff9}
\subsection{Residual analysis}
\label{sec:orgd93a1f3}
\subsubsection{Normality of residuals}
\label{sec:org52cb10e}

Figure \ref{fig:orgd8a1c69} illustrates QQ plot of the model residuals. The observer may say that the 
points exhibit a pattern that indicates that the residuals come from a distribution with heavier tails
than that of a normal distribution. 
\cite{Montgomery2012}. Still, the deviations from the diagonal line is relatively small, and hence
we conclude that the first Gauss-Markov condition is fulfilled. That is, the model errors seem to be 
normally distributed.

\begin{figure}[htbp]
\centering
\includegraphics[width=8cm]{/home/ludvigwgerdin/courses/Regression Analysis/regone/qqplot.png}
\caption{\label{fig:orgd8a1c69}
Normality plot of residuals.}
\end{figure}

\subsubsection{Fitted Against Residuals}
\label{sec:orgd6db02b}

Figure \ref{fig:org3d2c550} illustrates the fitted values \(\hat y_j\) against the R-student residuals. No apparent 
pattern is formed by the points, i.e. the points seem to be randomly scattered along the horizontal line.
Hence we conclude that the second Gauss-Markov condition is fulfilled, that is the errors have a constant 
variance.

\begin{figure}[htbp]
\centering
\includegraphics[width=8cm]{/home/ludvigwgerdin/courses/Regression Analysis/regone/far.png}
\caption{\label{fig:org3d2c550}
Fitted values against R-student residuals.}
\end{figure}

\subsubsection{Added Variable Analysis}
\label{sec:org81fcfc3}

Partial regression plots are found in figure \ref{fig:orgfc00962}, \ref{fig:orgff4a85f},
\ref{fig:org916f991}, and \ref{fig:org3c4c5d5}. All figures exhibits potential outliers 
(which will be further considered in section \ref{sec:org08cae6f}).
More specifically, in figure \ref{fig:orgfc00962} we note a 
few potential outliers on the right hand side of the plot for the \texttt{biceps} regressor, and on the
right and left hand side for the \texttt{forearm} regressor. Moreover, in figure \ref{fig:orgff4a85f}, we 
notice outliers on the right hand side of the \texttt{ankle} plot, and a group of potential outliers on the
\texttt{thigh} plot. Finally, we notice a few potential outliers in figure \ref{fig:org916f991} and 
\ref{fig:org3c4c5d5}.

Figure \ref{fig:orgff4a85f}, \ref{fig:org916f991}, and \ref{fig:org3c4c5d5} 
conveys important information about the information that \texttt{knee}, \texttt{height}, and
\texttt{chest} adds to the model. These regressors seem to follow a horizontal band along a fitted 
line from the origin, which may suggest that none of the regressors adds additional information 
to the predictions.

\begin{figure}[htbp]
\centering
\includegraphics[width=8cm]{/home/ludvigwgerdin/courses/Regression Analysis/regone/biceps_forearm_wrist_av.png}
\caption{\label{fig:orgfc00962}
Partial regression plots of regressors \texttt{biceps}, \texttt{forearm}, and \texttt{wrist}.}
\end{figure}   

\begin{figure}[htbp]
\centering
\includegraphics[width=8cm]{/home/ludvigwgerdin/courses/Regression Analysis/regone/thigh_knee_ankle_av.png}
\caption{\label{fig:orgff4a85f}
Partial regression plots of regressors \texttt{thigh}, \texttt{knee}, and \texttt{ankle}.}
\end{figure}

\begin{figure}[htbp]
\centering
\includegraphics[width=8cm]{/home/ludvigwgerdin/courses/Regression Analysis/regone/age_weight_height_neck_av.png}
\caption{\label{fig:org916f991}
Partial regression plots of regressors \texttt{age}, \texttt{weight}, \texttt{height}, and \texttt{neck}.}
\end{figure}

\begin{figure}[htbp]
\centering
\includegraphics[width=8cm]{/home/ludvigwgerdin/courses/Regression Analysis/regone/chest_abdomen_hip_av.png}
\caption{\label{fig:org3c4c5d5}
Partial regression plots of regressors \texttt{chest}, \texttt{abdomen}, and \texttt{hip}.}
\end{figure}
\subsection{Transformations of variables}
\label{sec:orgfb988d7}

Figure \ref{fig:orgea822ef} displays the values of \(\lambda\) to be used in a potential Box-Cox transformation of 
the dependent variable \texttt{density}. The \(\lambda\) that maximized the log-likelihood is 0.9 (0.7-1.1 95\% CI). 

Using \(\lambda = 0.9\) gives us the normal probability plot displayed on the right hand side in figure \ref{fig:orgea822ef}.
We notice that this affects the distribution of residuals by making it more light-tailed. That is, the 
the tails of the distribution are too light for the distribution to be considered normal.

In section \ref{sec:org3450992} we noted that there was no indication that a transformation was needed. 
Here, we see that the transformation of the response variable only makes matters worse.

\begin{figure}[h]
\centering
\includegraphics[width=8cm]{/home/ludvigwgerdin/courses/Regression Analysis/regone/boxcox_fit.png}
\caption{\label{fig:orgea822ef}
Values for lambda against the log-likelihood of \texttt{density} for Box-Cox transformations.}
\end{figure}

\subsection{Diagnostics and handling of Outliers}
\label{sec:org0398ef9}

Figure \ref{fig:org6979ff3} illustrates Cook's distance for all points, where the three observations with the largest 
Cook's distance are labelled. 

\begin{figure}[h]
\centering
\includegraphics[width=8cm]{/home/ludvigwgerdin/courses/Regression Analysis/regone/cd.png}
\caption{\label{fig:org6979ff3}
Plot of Cook's distance for all observations.}
\end{figure}

Figure \ref{fig:org135d653} reports the \(DFFITS\) values. We label observations as in figure \ref{fig:org6979ff3}. We observe 
that the three largest absolute \(DFFITS\) correspond to the same observations as in the Cook's distance plot.

Figure \ref{fig:org329c214}, \ref{fig:org08012db}, \ref{fig:orgfbb0b29}, and
\ref{fig:org1fc6e9a} presents \(DFBETA\) values for groups of regressors. Observation 39
is present in a number of these figures.
\begin{figure}[h]
\centering
\includegraphics[width=8cm]{/home/ludvigwgerdin/courses/Regression Analysis/regone/dffits.png}
\caption{\label{fig:org135d653}
\(DFFITS\) for all observations.}
\end{figure}

\begin{figure}[h]
\centering
\includegraphics[width=8cm]{/home/ludvigwgerdin/courses/Regression Analysis/regone/biceps_forearm_wrist_dfbeta.png}
\caption{\label{fig:org329c214}
\(DFBETA\) for regressors \texttt{biceps}, \texttt{forearm}, and \texttt{wrist}.}
\end{figure}

\begin{figure}[h]
\centering
\includegraphics[width=8cm]{/home/ludvigwgerdin/courses/Regression Analysis/regone/age_weight_height_neck_dfbeta.png}
\caption{\label{fig:org08012db}
\(DFBETA\) for regressors \texttt{age}, \texttt{weight}, \texttt{height} and \texttt{neck}.}
\end{figure}

\begin{figure}[h]
\centering
\includegraphics[width=8cm]{/home/ludvigwgerdin/courses/Regression Analysis/regone/thigh_knee_ankle_dfbeta.png}
\caption{\label{fig:orgfbb0b29}
\(DFBETA\) for regressors \texttt{thigh}, \texttt{knee}, and \texttt{ankle}.}
\end{figure}

\begin{figure}[h]
\centering
\includegraphics[width=8cm]{/home/ludvigwgerdin/courses/Regression Analysis/regone/chest_abdomen_hip_dfbeta.png}
\caption{\label{fig:org1fc6e9a}
\(DFBETA\) for regressors \texttt{chest}, \texttt{abdomen}, and \texttt{hip}.}
\end{figure}

\section{Conclusion}
\label{sec:orgf9d7f82}
\bibliographystyle{plain}
\bibliography{library}
\end{document}