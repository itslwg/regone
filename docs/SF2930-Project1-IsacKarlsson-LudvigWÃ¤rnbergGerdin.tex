% Created 2020-02-14 Fri 18:11
% Intended LaTeX compiler: pdflatex
\documentclass[11pt]{article}
\usepackage[utf8]{inputenc}
\usepackage[T1]{fontenc}
\usepackage{graphicx}
\usepackage{grffile}
\usepackage{longtable}
\usepackage{wrapfig}
\usepackage{rotating}
\usepackage[normalem]{ulem}
\usepackage{amsmath}
\usepackage{textcomp}
\usepackage{amssymb}
\usepackage{capt-of}
\usepackage{hyperref}
\usepackage{minted}
\usepackage[margin=1.25in]{geometry} \usepackage{booktabs} \usepackage{graphicx} \usepackage{adjustbox} \usepackage{amsmath} \hypersetup{colorlinks=true,linkcolor=blue} \usepackage{amsthm} \newtheorem{definition}{Definition} \usepackage{bookmark}
\author{Ludde}
\date{\today}
\title{}
\hypersetup{
 pdfauthor={Ludde},
 pdftitle={},
 pdfkeywords={},
 pdfsubject={},
 pdfcreator={Emacs 26.2 (Org mode 9.1.14)}, 
 pdflang={English}}
\begin{document}

\begin{titlepage}
\centering
\includegraphics[width=0.15\textwidth]{example-image-1x1}\par\vspace{1cm}
{\scshape\LARGE Kungliga Tekniska Högskolan \par}
\vspace{1cm}
{\scshape\Large SF2930 Regression Analysis \par}
\vspace{1.5cm}
{\huge\bfseries Report I \\  \par}
\vspace{2cm}
{\Large\itshape Isac Karlsson\\ Ludvig Wärnberg Gerdin}
\vfill
Examiner \par
\textsc{Tatjana Pavlenko}

\vfill

{\large \today\par}
\end{titlepage}

\newpage
\tableofcontents
\newpage

\section{Introduction and Project Goals}
\label{sec:orgfd704c2}
\section{Analyses and Model Development}
\label{sec:orgf4dc896}
\subsection{Residual analysis}
\label{sec:org978b79e}
\subsubsection{Normality of residuals}
\label{sec:org7242c9e}

The normality of residuals therefore ensures that the confidence intervals presented in section \ref{sec:org730b7bb}
are valid.

\subsubsection{Fitted Against Residuals}
\label{sec:orgb065d6b}
\subsubsection{Added Variable Analysis}
\label{sec:orgd033f63}
\subsection{Diagnostics and handling of Outliers}
\label{sec:org2ef79cf}
\subsection{Transformations of variables}
\label{sec:org558ce0e}
\subsection{Diagnostics and handling of Multicolinearity}
\label{sec:orge6de87f}
\section{Results}
\label{sec:org730b7bb}
\subsection{Residual analysis}
\label{sec:org24a5549}
\subsubsection{Normality of residuals}
\label{sec:orgc24be41}

Figure \ref{fig:orgb0e0070} illustrates QQ plot of the model residuals. The observer may say that the 
points exhibit a pattern that indicates that the residuals come from a distribution with heavier tails
than that of a normal distribution. 
\cite{Montgomery2012}. Still, the deviations from the diagonal line is relatively small, and hence
we conclude that the first Gauss-Markov condition is fulfilled. That is, the model errors seem to be 
normally distributed.

\begin{figure}[htbp]
\centering
\includegraphics[width=8cm]{/home/ludvigwgerdin/courses/Regression Analysis/regone/qqplot.png}
\caption{\label{fig:orgb0e0070}
Normality plot of residuals.}
\end{figure}

\subsubsection{Fitted Against Residuals}
\label{sec:org9954f69}

Figure \ref{fig:org6418d98} illustrates the fitted values \(\hat y_j\) against the R-student residuals. No apparent 
pattern is formed by the points, i.e. the points seem to be randomly scattered along the horizontal line.
Hence we conclude that the second Gauss-Markov condition is fulfilled, that is the errors have a constant 
variance.

\begin{figure}[htbp]
\centering
\includegraphics[width=8cm]{/home/ludvigwgerdin/courses/Regression Analysis/regone/far.png}
\caption{\label{fig:org6418d98}
Fitted values against R-student residuals.}
\end{figure}

\subsubsection{Added Variable Analysis}
\label{sec:org5420652}

Partial regression plots are found in figure \ref{fig:orgd5f74db}, \ref{fig:org77eabbe},
\ref{fig:org8318a5d}, and \ref{fig:org7f27407}. All figures exhibits potential outliers 
(which will be further considered in section \ref{sec:org2ef79cf}).
More specifically, in figure \ref{fig:orgd5f74db} we note a 
few potential outliers on the right hand side of the plot for the \texttt{biceps} regressor, and on the
right and left hand side for the \texttt{forearm} regressor. Moreover, in figure \ref{fig:org77eabbe}, we 
notice outliers on the right hand side of the \texttt{ankle} plot, and a group of potential outliers on the
\texttt{thigh} plot. Finally, we notice a few potential outliers in figure \ref{fig:org8318a5d} and 
\ref{fig:org7f27407}.

Figure \ref{fig:org77eabbe}, \ref{fig:org8318a5d}, and \ref{fig:org7f27407} 
conveys important information about the information that \texttt{knee}, \texttt{height}, and
\texttt{chest} adds to the model. These regressors seem to follow a horizontal band along a fitted 
line from the origin, which may suggest that none of the regressors adds additional information 
to the predictions.

\begin{figure}[htbp]
\centering
\includegraphics[width=8cm]{/home/ludvigwgerdin/courses/Regression Analysis/regone/biceps_forearm_wrist_av.png}
\caption{\label{fig:orgd5f74db}
Partial regression plots of regressors \texttt{biceps}, \texttt{forearm}, and \texttt{wrist}.}
\end{figure}   

\begin{figure}[htbp]
\centering
\includegraphics[width=8cm]{/home/ludvigwgerdin/courses/Regression Analysis/regone/thigh_knee_ankle_av.png}
\caption{\label{fig:org77eabbe}
Partial regression plots of regressors \texttt{thigh}, \texttt{knee}, and \texttt{ankle}.}
\end{figure}

\begin{figure}[htbp]
\centering
\includegraphics[width=8cm]{/home/ludvigwgerdin/courses/Regression Analysis/regone/age_weight_height_neck_av.png}
\caption{\label{fig:org8318a5d}
Partial regression plots of regressors \texttt{age}, \texttt{weight}, \texttt{height}, and \texttt{neck}.}
\end{figure}

\begin{figure}[htbp]
\centering
\includegraphics[width=8cm]{/home/ludvigwgerdin/courses/Regression Analysis/regone/chest_abdomen_hip_av.png}
\caption{\label{fig:org7f27407}
Partial regression plots of regressors \texttt{chest}, \texttt{abdomen}, and \texttt{hip}.}
\end{figure}
\subsection{Transformations of variables}
\label{sec:org110bf12}

Figure \ref{fig:orgc51dfd4} displays the values of \(\lambda\) to be used in a potential Box-Cox transformation of 
the dependent variable \texttt{density}. The \(\lambda\) that maximized the log-likelihood is 0.9 (0.7-1.1 95\% CI). 

Using \(\lambda = 0.9\) gives us the normal probability plot displayed on the right hand side in figure \ref{fig:orgc51dfd4}.
We notice that this affects the distribution of residuals by making it more light-tailed. That is, the 
the tails of the distribution are too light for the distribution to be considered normal.

In section \ref{sec:org978b79e} we noted that there was no indication that a transformation was needed. 
Here, we see that the transformation of the response variable only makes matters worse.

\begin{figure}[h]
\centering
\includegraphics[width=8cm]{/home/ludvigwgerdin/courses/Regression Analysis/regone/boxcox_fit.png}
\caption{\label{fig:orgc51dfd4}
Values for lambda against the log-likelihood of \texttt{density} for Box-Cox transformations.}
\end{figure}

\subsection{Diagnostics and handling of Outliers}
\label{sec:orgb3f92c2}

Figure \ref{fig:orgdb7ed08} illustrates Cook's distance for all points, where the three observations with the largest 
Cook's distance are labelled. Considering the cut-off \(D_i = 1\) as proposed in \cite{Montgomery2012}, 
where \(D_i\) is the Cook's distance for observation \(i\), we note that none of the observations would be 
considered influential. Still, observation 39 and 83 are largely different relative
to the other points in terms of their Cook's distance. 

\begin{figure}[h]
\centering
\includegraphics[width=8cm]{/home/ludvigwgerdin/courses/Regression Analysis/regone/cd.png}
\caption{\label{fig:orgdb7ed08}
Plot of Cook's distance for all observations.}
\end{figure}

Figure \ref{fig:org69597f5} reports the \(DFFITS\) values. We label observations as in figure \ref{fig:orgdb7ed08}. We observe 
that the three largest absolute \(DFFITS\) correspond to the same observations as in the Cook's distance plot.
The recommended cutoff-value referred to in \cite{Montgomery2012}, i.e. \(2\sqrt{\frac{p}{n}}\)
where \(p = 13\) is the number of potential regressors and \(n = 248\) is the sample size, is 
plotted as a dotted line, and the points that lie below or above this cut-off value is labelled.
We observe that several points are considered influential points when using that cut-off value.

Figure \ref{fig:org00f6d0e}, \ref{fig:org4c2a880}, \ref{fig:orgc73ecec}, and
\ref{fig:org734cdc2} presents \(DFBETA\) values for groups of regressors. Observation 39
is present in a number of these figures. Using the aforementioned cut-off value of \(\frac{2}{\sqrt{n}}\), we 
we note that none of these points would be considered influential points.
\begin{figure}[h]
\centering
\includegraphics[width=8cm]{/home/ludvigwgerdin/courses/Regression Analysis/regone/dffits.png}
\caption{\label{fig:org69597f5}
\(DFFITS\) for all observations.}
\end{figure}

\begin{figure}[h]
\centering
\includegraphics[width=8cm]{/home/ludvigwgerdin/courses/Regression Analysis/regone/biceps_forearm_wrist_dfbeta.png}
\caption{\label{fig:org00f6d0e}
\(DFBETA\) for regressors \texttt{biceps}, \texttt{forearm}, and \texttt{wrist}.}
\end{figure}

\begin{figure}[h]
\centering
\includegraphics[width=8cm]{/home/ludvigwgerdin/courses/Regression Analysis/regone/age_weight_height_neck_dfbeta.png}
\caption{\label{fig:org4c2a880}
\(DFBETA\) for regressors \texttt{age}, \texttt{weight}, \texttt{height} and \texttt{neck}.}
\end{figure}

\begin{figure}[h]
\centering
\includegraphics[width=8cm]{/home/ludvigwgerdin/courses/Regression Analysis/regone/thigh_knee_ankle_dfbeta.png}
\caption{\label{fig:orgc73ecec}
\(DFBETA\) for regressors \texttt{thigh}, \texttt{knee}, and \texttt{ankle}.}
\end{figure}

\begin{figure}[h]
\centering
\includegraphics[width=8cm]{/home/ludvigwgerdin/courses/Regression Analysis/regone/chest_abdomen_hip_dfbeta.png}
\caption{\label{fig:org734cdc2}
\(DFBETA\) for regressors \texttt{chest}, \texttt{abdomen}, and \texttt{hip}.}
\end{figure}

\section{Conclusion}
\label{sec:org06f9126}
\bibliographystyle{plain}
\bibliography{library}
\end{document}