% Created 2020-02-10 Mon 10:22
% Intended LaTeX compiler: pdflatex
\documentclass[11pt]{article}
\usepackage[utf8]{inputenc}
\usepackage[T1]{fontenc}
\usepackage{graphicx}
\usepackage{grffile}
\usepackage{longtable}
\usepackage{wrapfig}
\usepackage{rotating}
\usepackage[normalem]{ulem}
\usepackage{amsmath}
\usepackage{textcomp}
\usepackage{amssymb}
\usepackage{capt-of}
\usepackage{hyperref}
\usepackage{minted}
\usepackage[margin=1.25in]{geometry} \usepackage{booktabs} \usepackage{graphicx} \usepackage{adjustbox} \usepackage{amsmath} \hypersetup{colorlinks=true,linkcolor=blue} \usepackage{amsthm} \newtheorem{definition}{Definition} \usepackage{bookmark}
\author{Ludde}
\date{\today}
\title{}
\hypersetup{
 pdfauthor={Ludde},
 pdftitle={},
 pdfkeywords={},
 pdfsubject={},
 pdfcreator={Emacs 26.2 (Org mode 9.1.14)}, 
 pdflang={English}}
\begin{document}

\begin{titlepage}
\centering
\includegraphics[width=0.15\textwidth]{example-image-1x1}\par\vspace{1cm}
{\scshape\LARGE Kungliga Tekniska Högskolan \par}
\vspace{1cm}
{\scshape\Large SF2930 Regression Analysis \par}
\vspace{1.5cm}
{\huge\bfseries Report I \\  \par}
\vspace{2cm}
{\Large\itshape Isac Karlsson, 981105-6994\\ Ludvig Wärnberg Gerdin. 980411-6250}
\vfill
Examiner \par
\textsc{Tatjana Pavlenko}

\vfill

{\large \today\par}
\end{titlepage}

\newpage
\tableofcontents
\newpage

\section{Introduction and Project Goals}
\label{sec:org8daf345}
\section{Analyses and Model Development}
\label{sec:org8ccb171}
\subsection{Residual analysis}
\label{sec:orge9ec91e}
\subsubsection{Normality of residuals}
\label{sec:org15b15ab}
\subsubsection{Fitted Against Residuals}
\label{sec:orge354d77}
\subsubsection{Added Variable Analysis}
\label{sec:org3228fcf}
\subsection{Diagnostics and handling of Outliers}
\label{sec:orgdb6c46b}
\subsection{Transformations of variables}
\label{sec:org885c739}
\subsection{Diagnostics and handling of Multicolinearity}
\label{sec:orgf8a49dc}
\section{Results}
\label{sec:orgce9a94b}
\subsection{Residual analysis}
\label{sec:org9eec0f3}
\subsubsection{Normality of residuals}
\label{sec:org4c7f0a6}

Figure \ref{fig:org1d7b176} illustrates QQ plot of the model residuals. The observer may say that the 
points exhibit a pattern that indicates that the residuals come from a distribution with heavier tails
than that of a normal distribution. 
\cite{Montgomery2012}. Still, the deviations from the diagonal line is relatively small, and hence
we conclude that the first Gauss-Markov condition is fulfilled. That is, the model errors seem to be 
normally distributed.

\begin{figure}[htbp]
\centering
\includegraphics[width=8cm]{/home/ludvigwgerdin/courses/Regression Analysis/regone/qqplot.png}
\caption{\label{fig:org1d7b176}
Normality plot of residuals.}
\end{figure}

\subsubsection{Fitted Against Residuals}
\label{sec:org4e9f741}

Figure \ref{fig:orgb7be586} illustrates the fitted values \(\hat y_j\) against the R-student residuals. No apparent 
pattern is formed by the points, i.e. the points seem to be randomly scattered along the horizontal line.
Hence we conclude that the second Gauss-Markov condition is fulfilled, that is the errors have a constant 
variance.

\begin{figure}[htbp]
\centering
\includegraphics[width=8cm]{/home/ludvigwgerdin/courses/Regression Analysis/regone/far.png}
\caption{\label{fig:orgb7be586}
Fitted values against R-student residuals.}
\end{figure}

\subsubsection{Added Variable Analysis}
\label{sec:orgbec0ca3}

Partial regression plots are found in figure \ref{fig:org7a24822}, \ref{fig:orgadf4b23},
\ref{fig:org8cdfcb2}, and \ref{fig:org757b959}. All figures exhibits potential outliers 
(which will be further considered in section \ref{sec:orgdb6c46b}).
More specifically, in figure \ref{fig:org7a24822} we note a 
few potential outliers on the right hand side of the plot for the \texttt{biceps} regressor, and on the
right and left hand side for the \texttt{forearm} regressor. Moreover, in figure \ref{fig:orgadf4b23}, we 
notice outliers on the right hand side of the \texttt{ankle} plot, and a group of potential outliers on the
\texttt{thigh} plot. Finally, we notice a few potential outliers in figure \ref{fig:org8cdfcb2} and 
\ref{fig:org757b959}.

Figure \ref{fig:orgadf4b23}, \ref{fig:org8cdfcb2}, and \ref{fig:org757b959} 
conveys important information about the information that \texttt{knee}, \texttt{height}, and
\texttt{chest} adds to the model. These regressors seem to follow a horizontal band along a fitted 
line from the origin, which may suggest that none of the regressors adds additional information 
to the predictions.

\begin{figure}[htbp]
\centering
\includegraphics[width=8cm]{/home/ludvigwgerdin/courses/Regression Analysis/regone/biceps_forearm_wrist_av.png}
\caption{\label{fig:org7a24822}
Partial regression plots of regressors \texttt{biceps}, \texttt{forearm}, and \texttt{wrist}.}
\end{figure}   

\begin{figure}[htbp]
\centering
\includegraphics[width=8cm]{/home/ludvigwgerdin/courses/Regression Analysis/regone/thigh_knee_ankle_av.png}
\caption{\label{fig:orgadf4b23}
Partial regression plots of regressors \texttt{thigh}, \texttt{knee}, and \texttt{ankle}.}
\end{figure}

\begin{figure}[htbp]
\centering
\includegraphics[width=8cm]{/home/ludvigwgerdin/courses/Regression Analysis/regone/age_weight_height_neck_av.png}
\caption{\label{fig:org8cdfcb2}
Partial regression plots of regressors \texttt{age}, \texttt{weight}, \texttt{height}, and \texttt{neck}.}
\end{figure}

\begin{figure}[htbp]
\centering
\includegraphics[width=8cm]{/home/ludvigwgerdin/courses/Regression Analysis/regone/chest_abdomen_hip_av.png}
\caption{\label{fig:org757b959}
Partial regression plots of regressors \texttt{chest}, \texttt{abdomen}, and \texttt{hip}.}
\end{figure}
\subsection{Transformations of variables}
\label{sec:orgc09d491}

\begin{figure}[h]
\centering
\includegraphics[width=8cm]{/home/ludvigwgerdin/courses/Regression Analysis/regone/boxcox.png}
\caption{\label{fig:org564449d}
Values for lambda against the log-likelihood of \texttt{density} for Box-Cox transformations.}
\end{figure}

\subsection{Diagnostics and handling of Outliers}
\label{sec:org1b1baab}
\section{Conclusion}
\label{sec:orgc446a26}
\bibliographystyle{plain}
\bibliography{library}
\end{document}